%%%%%%%%%%%%%%%%%%%%%%%%%%%%%%%%%%%%%%
%% Document definition
%%%%%%%%%%%%%%%%%%%%%%%%%%%%%%%%%%%%%%

\documentclass[fleqn,letterpaper,11pt,oneside]{article}

% font selection
%----------------------
\usepackage{palatino}
%\usepackage{times}
%\usepackage{newcent}
%\usepackage{bookman}
%\usepackage{avant}
%\usepackage{helvet}

% Packages
%----------------------
\usepackage{graphicx}
\usepackage{amsmath}
\usepackage{fancyhdr}
\usepackage{enumerate}
\usepackage{indentfirst}
\usepackage{xspace}
\usepackage{array}
\usepackage{rotating}
\usepackage[
  pdfusetitle,
  pdfkeywords={minerva, software, inclusive},
  bookmarks,bookmarksopen,
  bookmarksnumbered,
  pdfpagelabels,
  colorlinks,linkcolor=black,citecolor=black,
  pdfview={Fit},
]
{hyperref}

% Need this for \toprule, etc.
\usepackage{booktabs}

% Add line numbers w/ \linenumbers.
\usepackage{lineno}

% Get basic author block formatting.
\usepackage{authblk}

% Multipage tables?
\usepackage{longtable}

% Page dimensions
%----------------------
\setlength{\hoffset}{0.05in}
\setlength{\oddsidemargin}{0.25in}
\setlength{\evensidemargin}{0.0in}
\setlength{\textwidth}{6.2in} 

\setlength{\voffset}{-0.25in}
\setlength{\topmargin}{0.0in}
\setlength{\headsep}{0.2in}
\setlength{\topskip}{0.0in}
\setlength{\headheight}{14.0pt}
\setlength{\footskip}{0.5in}
\setlength{\textheight}{8.9in}

\renewcommand{\baselinestretch}{1.6}
\setlength{\parindent}{0.25in}

% Define headers/footers
%----------------------
\pagestyle{fancy}
\fancyhead{}
\fancyfoot{}
\setlength{\marginparwidth}{0.25cm}
\setlength{\marginparsep}{0.1cm}
\addtolength{\headwidth}{\marginparsep}   
\addtolength{\headwidth}{\marginparwidth}  

% Experiment
%------------
\newcommand{\minerva}{MINER$\nu$A\xspace}

% General Math
%-------------------
\newcommand{\be}{\begin{equation}}
\newcommand{\ee}{\end{equation}}
\newcommand{\bea}{\begin{eqnarray}}
\newcommand{\eea}{\end{eqnarray}}
\newcommand{\abs}[1]{\mathopen| #1 \mathclose|}
\newcommand{\bra}[1]{\langle #1|}
\newcommand{\ket}[1]{|#1\rangle}
\newcommand{\braket}[2]{\langle #1|#2\rangle}

% Theorems
%  Number lemmas within theorems, http://www.mackichan.com/index.html?techtalk/611.htm~mainFrame
%-------------------------------------------------------------------------------------------------
\newtheorem{definition}{Definition}[section]
\newtheorem{theorem}{Theorem}[section]
\newtheorem{lemma}{Lemma}
\numberwithin{lemma}{theorem}
\newtheorem{proposition}[theorem]{Proposition}
\newtheorem{corollary}[theorem]{Corollary}

% Formatting
%  Overbar idea from:
%   http://tex.stackexchange.com/questions/22100/the-bar-and-overline-commands
%-------------------------------------------------------------------------------
\newcommand{\overbar}[1]{\mkern 1.5mu\overline{\mkern-1.5mu#1\mkern-1.5mu}\mkern 1.5mu}

% Places
%---------
\newcommand{\FNAL}{Fermi National Accelerator Laboratory, Batavia, Illinois 60510, USA}
\newcommand{\Rochester}{University of Rochester, Rochester, New York 14610 USA}


% Examples:
%
% Matrix
%--------
% A = \begin{pmatrix} A_{11} & A_{12} \\ A_{21} & A_{22} \end{pmatrix}
%
% Table
%----------
%\begin{table}[htb]
%\centering
%\begin{tabular}{crr}
%\toprule
%$X$: & 1 & 2 \\
%\midrule
%$Y$: & 2 & 3 \\
%\bottomrule
%\end{tabular}
%\caption{A table}
%\label{tbl:atable}
%\end{table}
%
% Figure
%----------
%\begin{figure}
%  \centering
%  \includegraphics[height=0.32\textheight]{RecursionTree}
%  \caption{A caption.}
%  \label{fig:recursiontree}
%\end{figure}
%
% Braced Equation
%------------------
%\[
%  f(n) = \left\{ 
%  \begin{array}{l l}
%    n/2 & \quad \text{if $n$ is even}\\
%    -(n+1)/2 & \quad \text{if $n$ is odd}\\
%  \end{array} \right.
%\]




\begin{document}

\linenumbers

\title{The Readout Sequence for the \minerva CROC}

\author[1]{G.~N.~Perdue}
\affil[1]{\Rochester}

\maketitle

\abstract{We describe the readout sequence commands for the \minerva CROC.}

\section{Introduction}
\label{sec:introduction}

This document describes the commands required to readout the \minerva CROC \cite{refcroc}.

\section{Reading and Clearing the CROC Status Register}
\label{sec:readandclearcrocstatus}

A common first step in readout is to clear the status register and reset the Dual Port Memory (DPM) pointer.

\begin{enumerate}
\item Reset the DPM: VME WriteCycle, sending the CROC DPM Reset Value to the CROC Base + Clear Status (\hex{2030}) + Channel Offset Address ($i \times \hex{4000}$, where $i$ indexes the \emph{chain} from 0-3). The DPM Reset Value is an array of bytes (unsigned char) with value $\left\{\hex{08},\hex{08}\right\}$. 
\item Clear the Status Register: VME WriteCycle, sending the CROC Channel Reset Value to the CROC Base + Clear Status (\hex{2030}) + Channel Offset Address ($i \times \hex{4000}$, where $i$ indexes the \emph{chain} from 0-3). The DPM Reset Value is an array of bytes (unsigned char) with value $\left\{\hex{02},\hex{02}\right\}$. 
\item Note that the previous two steps can be combined by sending the message $\left\{\hex{0A},\hex{0A}\right\}$ to reset the DPM pointer and channel status simultaneously.
\item Read the Status: VME Read Cycle for address CROC Base + Status (\hex{2020}) + Channel Offset Address ($i \times \hex{4000}$). A status value of \hex{3703} indicates perfect health.
\end{enumerate}


\section{Reading Data from the FEBs}
\label{sec:readfromfebs}

\begin{enumerate}
\item First, we must clear the status register and reset the Dual Port Memory (DPM) pointer. See Section \ref{sec:readandclearcrocstatus}.
\item Next, we may read data from the FEBs.
\begin{enumerate}
\item First, we must compose a message frame.
\item Then, we write the frame to the CROC Channel Channel FIFO Input register: VME Write Cycle to CROC Base + FIFO Input (\hex{2000}) + Channel Offset Address ($i \times \hex{4000}$).
\item Once we have written the message, we send the Send Message command to the CROC: VME Write Cycle Send Message $\left\{\hex{01},\hex{01}\right\}$ to CROC Base + CROC Send Message (\hex{2010}) + Channel Offset Address ($i \times \hex{4000}$).
\item While the message is being processed, read the CROC status register and check for errors. We want to find \hex{3703} where the last byte encodes that the message has been successfully sent and received.
\item Once the message has been sent and received, read the DPM Pointer to see if any data is available: VME Read Cycle, reading CROC Base + CROC DPM Pointer (\hex{2050}) + Channel Offset Address ($i \times \hex{4000}$).
\item Finally, if the DPM pointer length indicates we have data to read, execute a VME Read BLT for an array of size equal to the pointer length at CROC Base + CROC Memory (\hex{0000}) + Channel Offset Address ($i \times \hex{4000}$).
\end{enumerate}
\end{enumerate}

\section{Writing Data to the FEBs}
\label{sec:writetofebs}

\begin{enumerate}
\item 
\item 
\end{enumerate}


\bibliographystyle{plain}
\bibliography{DAQElectronics}

\end{document}
