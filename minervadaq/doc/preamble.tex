%%%%%%%%%%%%%%%%%%%%%%%%%%%%%%%%%%%%%%
%% Document definition
%%%%%%%%%%%%%%%%%%%%%%%%%%%%%%%%%%%%%%

\documentclass[fleqn,letterpaper,11pt,oneside]{article}

% font selection
%----------------------
\usepackage{palatino}
%\usepackage{times}
%\usepackage{newcent}
%\usepackage{bookman}
%\usepackage{avant}
%\usepackage{helvet}

% Packages
%----------------------
\usepackage{graphicx}
\usepackage{amsmath}
\usepackage{fancyhdr}
\usepackage{enumerate}
\usepackage{indentfirst}
\usepackage{xspace}
\usepackage{array}
\usepackage{rotating}
\usepackage[
  pdfusetitle,
  pdfkeywords={minerva, software, inclusive},
  bookmarks,bookmarksopen,
  bookmarksnumbered,
  pdfpagelabels,
  colorlinks,linkcolor=black,citecolor=black,
  pdfview={Fit},
]
{hyperref}

% Need this for \toprule, etc.
\usepackage{booktabs}

% Add line numbers w/ \linenumbers.
\usepackage{lineno}

% Get basic author block formatting.
\usepackage{authblk}

% Multipage tables?
\usepackage{longtable}

% Page dimensions
%----------------------
\setlength{\hoffset}{0.05in}
\setlength{\oddsidemargin}{0.25in}
\setlength{\evensidemargin}{0.0in}
\setlength{\textwidth}{6.2in} 

\setlength{\voffset}{-0.25in}
\setlength{\topmargin}{0.0in}
\setlength{\headsep}{0.2in}
\setlength{\topskip}{0.0in}
\setlength{\headheight}{14.0pt}
\setlength{\footskip}{0.5in}
\setlength{\textheight}{8.9in}

\renewcommand{\baselinestretch}{1.6}
\setlength{\parindent}{0.25in}

% Define headers/footers
%----------------------
\pagestyle{fancy}
\fancyhead{}
\fancyfoot{}
\setlength{\marginparwidth}{0.25cm}
\setlength{\marginparsep}{0.1cm}
\addtolength{\headwidth}{\marginparsep}   
\addtolength{\headwidth}{\marginparwidth}  

% Experiment
%------------
\newcommand{\minerva}{MINER$\nu$A\xspace}

% General Math
%-------------------
\newcommand{\be}{\begin{equation}}
\newcommand{\ee}{\end{equation}}
\newcommand{\bea}{\begin{eqnarray}}
\newcommand{\eea}{\end{eqnarray}}
\newcommand{\abs}[1]{\mathopen| #1 \mathclose|}
\newcommand{\bra}[1]{\langle #1|}
\newcommand{\ket}[1]{|#1\rangle}
\newcommand{\braket}[2]{\langle #1|#2\rangle}

% Theorems
%  Number lemmas within theorems, http://www.mackichan.com/index.html?techtalk/611.htm~mainFrame
%-------------------------------------------------------------------------------------------------
\newtheorem{definition}{Definition}[section]
\newtheorem{theorem}{Theorem}[section]
\newtheorem{lemma}{Lemma}
\numberwithin{lemma}{theorem}
\newtheorem{proposition}[theorem]{Proposition}
\newtheorem{corollary}[theorem]{Corollary}

% Formatting
%  Overbar idea from:
%   http://tex.stackexchange.com/questions/22100/the-bar-and-overline-commands
%-------------------------------------------------------------------------------
\newcommand{\overbar}[1]{\mkern 1.5mu\overline{\mkern-1.5mu#1\mkern-1.5mu}\mkern 1.5mu}

% Places
%---------
\newcommand{\FNAL}{Fermi National Accelerator Laboratory, Batavia, Illinois 60510, USA}
\newcommand{\Rochester}{University of Rochester, Rochester, New York 14610 USA}


% Examples:
%
% Matrix
%--------
% A = \begin{pmatrix} A_{11} & A_{12} \\ A_{21} & A_{22} \end{pmatrix}
%
% Table
%----------
%\begin{table}[htb]
%\centering
%\begin{tabular}{crr}
%\toprule
%$X$: & 1 & 2 \\
%\midrule
%$Y$: & 2 & 3 \\
%\bottomrule
%\end{tabular}
%\caption{A table}
%\label{tbl:atable}
%\end{table}
%
% Figure
%----------
%\begin{figure}
%  \centering
%  \includegraphics[height=0.32\textheight]{RecursionTree}
%  \caption{A caption.}
%  \label{fig:recursiontree}
%\end{figure}
%
% Braced Equation
%------------------
%\[
%  f(n) = \left\{ 
%  \begin{array}{l l}
%    n/2 & \quad \text{if $n$ is even}\\
%    -(n+1)/2 & \quad \text{if $n$ is odd}\\
%  \end{array} \right.
%\]

