%%%%%%%%%%%%%%%%%%%%%%%%%%%%%%%%%%%%%%
%% Document definition
%%%%%%%%%%%%%%%%%%%%%%%%%%%%%%%%%%%%%%

\documentclass[fleqn,letterpaper,11pt,oneside]{article}

% font selection
%----------------------
\usepackage{palatino}
%\usepackage{times}
%\usepackage{newcent}
%\usepackage{bookman}
%\usepackage{avant}
%\usepackage{helvet}

% Packages
%----------------------
\usepackage{graphicx}
\usepackage{amsmath}
\usepackage{fancyhdr}
\usepackage{enumerate}
\usepackage{indentfirst}
\usepackage{xspace}
\usepackage{array}
\usepackage{rotating}
\usepackage[
  pdfusetitle,
  pdfkeywords={minerva, software, inclusive},
  bookmarks,bookmarksopen,
  bookmarksnumbered,
  pdfpagelabels,
  colorlinks,linkcolor=black,citecolor=black,
  pdfview={Fit},
]
{hyperref}

% Need this for \toprule, etc.
\usepackage{booktabs}

% Add line numbers w/ \linenumbers.
\usepackage{lineno}

% Get basic author block formatting.
\usepackage{authblk}

% Multipage tables?
\usepackage{longtable}

% Page dimensions
%----------------------
\setlength{\hoffset}{0.05in}
\setlength{\oddsidemargin}{0.25in}
\setlength{\evensidemargin}{0.0in}
\setlength{\textwidth}{6.2in} 

\setlength{\voffset}{-0.25in}
\setlength{\topmargin}{0.0in}
\setlength{\headsep}{0.2in}
\setlength{\topskip}{0.0in}
\setlength{\headheight}{14.0pt}
\setlength{\footskip}{0.5in}
\setlength{\textheight}{8.9in}

\renewcommand{\baselinestretch}{1.6}
\setlength{\parindent}{0.25in}

% Define headers/footers
%----------------------
\pagestyle{fancy}
\fancyhead{}
\fancyfoot{}
\setlength{\marginparwidth}{0.25cm}
\setlength{\marginparsep}{0.1cm}
\addtolength{\headwidth}{\marginparsep}   
\addtolength{\headwidth}{\marginparwidth}  

% Experiment
%------------
\newcommand{\minerva}{MINER$\nu$A\xspace}

% General Math
%-------------------
\newcommand{\be}{\begin{equation}}
\newcommand{\ee}{\end{equation}}
\newcommand{\bea}{\begin{eqnarray}}
\newcommand{\eea}{\end{eqnarray}}
\newcommand{\abs}[1]{\mathopen| #1 \mathclose|}
\newcommand{\bra}[1]{\langle #1|}
\newcommand{\ket}[1]{|#1\rangle}
\newcommand{\braket}[2]{\langle #1|#2\rangle}

% Theorems
%  Number lemmas within theorems, http://www.mackichan.com/index.html?techtalk/611.htm~mainFrame
%-------------------------------------------------------------------------------------------------
\newtheorem{definition}{Definition}[section]
\newtheorem{theorem}{Theorem}[section]
\newtheorem{lemma}{Lemma}
\numberwithin{lemma}{theorem}
\newtheorem{proposition}[theorem]{Proposition}
\newtheorem{corollary}[theorem]{Corollary}

% Formatting
%  Overbar idea from:
%   http://tex.stackexchange.com/questions/22100/the-bar-and-overline-commands
%-------------------------------------------------------------------------------
\newcommand{\overbar}[1]{\mkern 1.5mu\overline{\mkern-1.5mu#1\mkern-1.5mu}\mkern 1.5mu}

% Places
%---------
\newcommand{\FNAL}{Fermi National Accelerator Laboratory, Batavia, Illinois 60510, USA}
\newcommand{\Rochester}{University of Rochester, Rochester, New York 14610 USA}


% Examples:
%
% Matrix
%--------
% A = \begin{pmatrix} A_{11} & A_{12} \\ A_{21} & A_{22} \end{pmatrix}
%
% Table
%----------
%\begin{table}[htb]
%\centering
%\begin{tabular}{crr}
%\toprule
%$X$: & 1 & 2 \\
%\midrule
%$Y$: & 2 & 3 \\
%\bottomrule
%\end{tabular}
%\caption{A table}
%\label{tbl:atable}
%\end{table}
%
% Figure
%----------
%\begin{figure}
%  \centering
%  \includegraphics[height=0.32\textheight]{RecursionTree}
%  \caption{A caption.}
%  \label{fig:recursiontree}
%\end{figure}
%
% Braced Equation
%------------------
%\[
%  f(n) = \left\{ 
%  \begin{array}{l l}
%    n/2 & \quad \text{if $n$ is even}\\
%    -(n+1)/2 & \quad \text{if $n$ is odd}\\
%  \end{array} \right.
%\]




\begin{document}

\linenumbers

\title{The Readout Sequence for the \minerva CROC-E}

\author[1]{G.~N.~Perdue}
\affil[1]{\Rochester}

\maketitle

\abstract{We describe the readout sequence commands for the \minerva CROC-E.}

\section{Introduction}
\label{sec:introduction}

This document describes the commands required to readout the \minerva CROC-E.
% \cite{refcroce} - No DocDB available yet.
We assume there is interesting data sitting on the FEB in this document.

\section{CROC+ Mode: Sequencer Readout Disabled}
\label{sec:seqreaddisab}

Tentative sequence (assuming one FEB); all addresses assume CROCE Base + Channel Offset Address ($i \times \hex{4000}$):
\begin{enumerate}
\item Configuration (\hex{28002}): \hex{0001} to disable the Sequencer Readout mode and set the number of FEBs to one.
\item Command (\hex{2804}): \hex{8000} to clear the status register.
\item Read the Frame (\hex{28020}) and Tx/Rx (\hex{28040}) Status registers? Or only one?
\item Write the frame we would like to send/read to the Send Memory Register (\hex{22000}). 
\item Write send message (\hex{4000}) to the Command Register (\hex{28004}) to send the message.
\item Poll the status register looking for Sequencer Done and Frame Received bits to go to one (mask for those two bits: \hex{0410}). Once both bits are one, we may begin readout. What value of the register constitutes perfect health?
\item Read the Receive Memory Pointer (\hex{28080}) to get the amount of data.
\item Read the Receive Memory Register (\hex{00000}), using the result from the pointer read, to get that many bytes. 
\item At the end of readout, optionally read the Event Counter register (\hex{28008}).
\end{enumerate}

\section{Comments on the CROC+ Mode: Sequencer Readout Disabled}

It is sufficient to send the ClearStatus command \hex{8000}. ClearStatus will reset almost ALL LOGIC inside the channel's FPGA, including, the send and receive memory write pointers. The ResetSendMemoryPointer command (0x2000) is active only when the SendMemoryWriteMode is set to FIFOMode (default). It is a distinct command to allow the flexibility to reset only this pointer, WITHOUT resetting the ReceiveMemWritePointer. This way one may write / send / receive a new frame(s) and have them ALL stored, one after the other, in ReceiveMemory. Otherwise, if you do ClearStatus then ALL incoming frames will overwrite each other since they will be written starting from address zero of the ReceiveMemory. Similarly the ResetReceiveMemoryWritePointer \hex{1000} gives you the flexibility to reset ONLY this pointer.

\section{CROC+ Mode: Sequencer Readout Enabled}
\label{sec:seqreadenab}

Tentative sequence (assuming one FEB); all addresses assume CROCE Base + Channel Offset Address ($i \times \hex{4000}$):
\begin{enumerate}
\item Configuration (\hex{28002}): \hex{8001} to enable the Sequencer Readout mode and set the number of FEBs to one.
\item Command (\hex{2804}): \hex{B000} to clear the status register and reset both the send and receive memory pointers.
\item Read the Frame (\hex{28020}) and Tx/Rx (\hex{28040}) Status registers? Or only one?
\item Send an RDFE signal.
\item Poll the status register looking for Sequencer Done and Frame Received bits to go to one (mask for those two bits: \hex{0410}). Once both bits are one, we may begin readout.
\item Read the Receive Memory Pointer (\hex{28080}) to get the amount of data.
\item Read the Receive Memory Register (\hex{00000}), using the result from the pointer read, to get that many bytes. In this case, we have a glob with all the frame data from the FEB.
\item At the end of readout, optionally read the Event Counter register (\hex{28008}).
\end{enumerate}

\section{CROC+ Mode: Reading and Clearing the CROC Status Register}
\label{sec:readandclearcrocstatus}



\section{CROC+ Mode: Reading Data from the FEBs}
\label{sec:readfromfebs}


\section{CROC+ Mode: Writing Data to the FEBs}
\label{sec:writetofebs}


\bibliographystyle{plain}
\bibliography{DAQElectronics}

\end{document}
